\section{Introducción}
\label{ch:introduccion}

La introducción es la primera sección sustantiva de la tesis y tiene como propósito principal guiar al lector a través del contexto y la relevancia del estudio. En ella, se presenta el problema de investigación, se establecen los objetivos, se justifica la realización del trabajo y se delinean los alcances y limitaciones. Debe ser escrita de forma clara y concisa, capturando el interés del lector y proporcionando una visión general de lo que se abordará en los capítulos subsiguientes.

\subsection{Antecedentes}
\label{sec:antecedentes}

En esta sección, se presenta una revisión concisa y pertinente de investigaciones previas, literatura existente y el estado del arte relacionado con el tema de la tesis. El objetivo es situar el trabajo actual dentro del conocimiento acumulado, identificar vacíos en la investigación y demostrar la originalidad y la necesidad del estudio propuesto. Se deben citar adecuadamente las fuentes, mostrando cómo el trabajo se construye sobre cimientos previos o cómo aborda aspectos no resueltos.

\subsection{Definición del Problema}
\label{sec:problema}

Aquí se describe de manera clara y precisa el problema que la tesis busca resolver. Es fundamental articular el problema de forma que sea comprensible, relevante y factible de investigar. Esta sección debe responder a preguntas como: ¿Qué situación actual es insatisfactoria o genera una necesidad? ¿Qué aspectos de este problema no han sido abordados adecuadamente por investigaciones anteriores? La definición del problema puede ser formulada como una pregunta de investigación o una declaración clara de la brecha de conocimiento.

\subsection{Objetivos}
\label{sec:objetivos}

Los objetivos son las metas que se esperan alcanzar con la realización de la investigación. Deben ser claros, medibles, alcanzables, relevantes y con un tiempo definido (SMART, por sus siglas en inglés). Se dividen en un objetivo general y varios objetivos específicos.

\subsubsection{Objetivo General}
\label{subsec:objetivo_general}

El objetivo general expresa el propósito fundamental del estudio en términos amplios. Es la meta principal que se busca lograr y generalmente abarca la totalidad de la investigación. Debe ser coherente con la definición del problema.
\begin{itemize}
    \item \textbf{Ejemplo:} Desarrollar un sistema automatizado para la detección temprana de fallas en turbinas eólicas utilizando técnicas de aprendizaje automático.
\end{itemize}

\subsubsection{Objetivos Específicos}
\label{subsec:objetivos_especificos}

Los objetivos específicos detallan los pasos o las acciones concretas que se llevarán a cabo para alcanzar el objetivo general. Son tareas más pequeñas y desglosadas que, al ser cumplidas, permiten la consecución del objetivo principal.
\begin{itemize}
    \item \textbf{Ejemplo 1:} Recopilar y preprocesar un conjunto de datos de vibraciones y parámetros operativos de turbinas eólicas.
    \item \textbf{Ejemplo 2:} Implementar y evaluar diferentes algoritmos de aprendizaje automático para la clasificación de estados operativos (normal, pre-falla, falla).
    \item \textbf{Ejemplo 3:} Diseñar una interfaz de usuario para la visualización de alertas y el monitoreo en tiempo real del estado de las turbinas.
\end{itemize}

\subsection{Justificación}
\label{sec:justificacion}

La justificación explica por qué es importante y relevante llevar a cabo la investigación. Se deben argumentar los beneficios y la utilidad del estudio desde diferentes perspectivas: teórica, práctica, social, económica, tecnológica, etc. Responde a la pregunta: ¿Por qué vale la pena invertir tiempo y recursos en esta investigación? Aquí se resaltan los aportes que la tesis generará al campo del conocimiento o a la solución de problemas específicos.

\subsection{Alcances y Limitaciones}
\label{sec:alcances_limitaciones}

\subsubsection{Alcances}
\label{subsec:alcances}

Los alcances definen claramente hasta dónde llegará el estudio, qué aspectos cubrirá y cuáles serán sus fronteras. Establecen qué elementos, variables, poblaciones o contextos serán incluidos en la investigación. Es esencial para evitar expectativas irrealistas y para mantener el enfoque del trabajo.

\subsubsection{Limitaciones}
\label{subsec:limitaciones}

Las limitaciones son los factores externos o internos que podrían restringir la validez, generalización o profundidad de los resultados de la investigación. Es importante identificarlas honestamente, ya que demuestran un pensamiento crítico por parte del investigador. Pueden incluir restricciones de tiempo, recursos, acceso a datos, metodológicas, etc.

\subsection{Ejemplos de Tablas e Inclusión de Imágenes}
\label{sec:ejemplos_elementos}

Para ilustrar el uso de elementos visuales en su tesis, a continuación se presentan ejemplos básicos de cómo incluir tablas e imágenes en LaTeX.

\subsubsection{Inclusión de Tablas}

Las tablas son herramientas fundamentales para presentar datos de manera organizada y comprensible.
\begin{table}[H]
    \centering
    \caption{Clasificación de Sensores Comunes en Ingeniería.}
    \label{tab:sensores_comunes}
    \begin{tabular}{|l|c|c|}
        \hline
        \textbf{Tipo de Sensor} & \textbf{Magnitud Medida} & \textbf{Principio de Funcionamiento} \\
        \hline
        Termopar & Temperatura & Efecto Seebeck \\
        Extensómetro & Deformación & Resistencia eléctrica \\
        Acelerómetro & Aceleración & Efecto piezoeléctrico \\
        Encoder rotatorio & Posición angular & Óptico/Magnético \\
        \hline
    \end{tabular}
    \caption*{Nota: Esta tabla es un ejemplo para demostrar la estructura básica de una tabla en LaTeX.}
\end{table}
Como se puede observar en la Tabla \ref{tab:sensores_comunes}, la información se presenta de forma clara y estructurada. Es importante añadir un \verb|`\caption`|  para describir la tabla y un  \verb|`\label`|  para referenciarla en el texto.

\subsubsection{Inclusión de Imágenes}

Las imágenes, gráficos o diagramas son cruciales para complementar la explicación textual y facilitar la comprensión de conceptos complejos o resultados visuales.
\begin{figure}[H]
    \centering
    \includegraphics[width=0.7\textwidth]{img/diagrama_proceso}
    \caption{Diagrama de un proceso de control automático.}
    \label{fig:diagrama_control}
\end{figure}
% La Figura \ref{fig:diagrama_control} ilustra un ejemplo de un diagrama de un proceso de control. Para incluir una imagen, se utiliza el entorno `figure` y el comando `\includegraphics`. Es fundamental especificar la ruta de la imagen (en este caso, `images/diagrama_proceso.png`) y un `width` para controlar su tamaño. Al igual que con las tablas, un `\caption` y un `\label` son indispensables. Es crucial que cada imagen tenga una calidad adecuada y que sea relevante para el texto.

\begin{figure}[H]
    \centering
    \begin{subfigure}[b]{0.45\textwidth}
        \includegraphics[width=\textwidth]{img/grafico_datos1}
        \caption{Datos de Temperatura.}
        \label{fig:subfig_temp}
    \end{subfigure}
    \hfill
    \begin{subfigure}[b]{0.45\textwidth}
        \includegraphics[width=\textwidth]{img/grafico_datos2}
        \caption{Datos de Presión.}
        \label{fig:subfig_pres}
    \end{subfigure}
    \caption{Comparación de datos operativos de un sistema (Temperatura vs. Presión).}
    \label{fig:comparacion_datos}
\end{figure}
Incluso es posible combinar varias imágenes en una sola figura, como se muestra en la Figura \ref{fig:comparacion_datos}, utilizando el paquete `subcaption`.

Asegúrense de que todas las figuras y tablas estén debidamente referenciadas en el texto y que su contenido sea autoexplicativo, utilizando los pies de figura y títulos de tabla de manera efectiva.