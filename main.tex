% Actualizacion 04/09/2025
% Autor: Said Polanco-Martagón


\documentclass[12pt]{article}
\usepackage[left=2.5cm,top=2.0cm,right=2.5cm,bottom=3.0cm]{geometry}
\usepackage[utf8]{inputenc}
\usepackage[spanish,mexico]{babel}
%\usepackage[english, american]{babel}
%\usepackage[spanish,es-tabla]{babel}
\usepackage[linguistics]{forest}
\usepackage{amssymb, amsmath, amsbsy} % simbolitos
\usepackage{longtable} % para tablas largas
\usepackage{graphicx}
\usepackage{fancyhdr}
\usepackage{xcolor}
\usepackage{multirow}
\usepackage{listings}
\usepackage{caption}
\usepackage{subcaption}
%\usepackage{parskip}
\usepackage[skip=12pt plus1pt]{parskip}
\usepackage{pdfpages} % Incluir PDF en documento en LATEX
\usepackage{verbatim} % comentarios
\usepackage{algpseudocode}
\usepackage{algorithm}
\usepackage{pdflscape}
\usepackage{multirow}
\usepackage{afterpage}
\usepackage{array,booktabs,ragged2e}
%\newcolumntype{R}[1]{>{\RaggedLeft\arraybackslash}p{#1}}
\newcolumntype{L}[1]{>{\raggedright\let\newline\\\arraybackslash\hspace{0pt}}m{#1}}
\newcolumntype{C}[1]{>{\centering\let\newline\\\arraybackslash\hspace{0pt}}m{#1}}
\newcolumntype{R}[1]{>{\raggedleft\let\newline\\\arraybackslash\hspace{0pt}}m{#1}}

\floatname{algorithm}{Algoritmo}
\renewcommand{\listalgorithmname}{Lista de algoritmos}
\renewcommand{\algorithmicrequire}{\textbf{Entrada:}}
\renewcommand{\algorithmicensure}{\textbf{Salida:}}

% Comentario con respecto a las referencias:
% Por defecto este documento utiliza el formato IEEEtr para formatear las referencias. Si algun asesor requiere formato APA, solo comente la siguiente linea y descomente la linea debajo. Tambien para que las citas funcionen de manera adecuada, el paquete label debe tener las opciones mostradas, o en su defecto, el año en el listado no lo muestra correctamente \usepackage[english, american]{babel}

% Para utilizar el formato de citas IEEE y comentar los dos parrafos siguientes
\usepackage[backend=bibtex,sorting=none]{biblatex}
% Para utilizar el formato APA, sugiero comentar la linea anterior y descomentar las dos proximas lineas
%\usepackage[backend=biber,style=apa]{biblatex}
%\DeclareLanguageMapping{english}{american-apa}

\makeatletter
\DefineBibliographyExtras{spanish}{%
  \setcounter{smartand}{1}%
  \let\lbx@finalnamedelim=\lbx@es@smartand
  \let\lbx@finallistdelim=\lbx@es@smartand
}
\renewbibmacro*{name:delim:apa:family-given}[1]{%
  \ifnumgreater{\value{listcount}}{\value{liststart}}
    {\ifboolexpr{
       test {\ifnumless{\value{listcount}}{\value{liststop}}}
       or
       test \ifmorenames
     }
       {\printdelim{multinamedelim}}
       {\lbx@finalnamedelim{#1}}}
    {}}
\makeatother


% Estas lineas permiten romper los hipervinculos muy largos en las referencias!!!!
\setcounter{biburllcpenalty}{7000}
\setcounter{biburlucpenalty}{8000}
\addbibresource{x.bib} % ARCHIVO DE BIBLIOGRAFÍA


%\usepackage{url}
\usepackage[bookmarks=true,breaklinks=true,bookmarksopen=false,colorlinks=true,linkcolor=blue]{hyperref}
\usepackage[hyphenbreaks]{breakurl}
% Regla que define explicitamente que caracteres rompen los hipervinculos para separar las lineas
%https://es.overleaf.com/11089898rhgykrqyqytx
% Actualiza en automático la fecha de las citas de internet a la fecha de la compilación del documento
\usepackage{datetime}
\newdateformat{specialdate}{\twodigit{\THEDAY}-\twodigit{\THEMONTH}-\THEYEAR}
%\newdateformat{specialdate}{\twodigit{\THEDAY}-\THEYEAR}
\date{\specialdate\today}

\newcommand{\HRule}{\rule{\linewidth}{0.25mm}}


% CONSTANTES NECESARIAS PARA EL DOCUMENTO ---> MODIFIQUEN A SU CRITERIO
\newcommand{\ncarrera}            {Ingeniería en Tecnologías de la Información}
\newcommand{\nasesorinstitucional}{Dr. Marco Aurelio Nuño Maganda}
\newcommand{\NombreAlumno}{Luis Gerardo Perales Torres}
%Hombres cambien LA por EL
\newcommand{\elolaNombreAlumno}{el}  
\newcommand{\OA}           {o}  %Hombres cambien A por O
\newcommand{\Matricula}           {1730505}  %SU MATRICULA
\newcommand{\NombreProyecto}{Notificación de actualizaciones de la información en un sistema de información web de pacientes hospitalizados a sus familiares del Sagrado Corazón de Jesus.}
\newcommand{\fechacarta}{26 de Abril de 2021}
\newcommand{\ncuatrimestre}{Mayo-agosto 2021}
\newcommand{\nevalador}{Dr. Hiram Herrera Rivas}
\newcommand{\FechaExposicion}{11 de Agosto de 2021}
\newcommand{\HoraExposionFormatoVenticuatroHoras}{10:00}
\newcommand{\elolaNombreEmpresa}{la}  %Si la empresa es femenino (por ejemplo universidad, usen la) o masculino (el instituto) pongan el
\newcommand{\organismoreceptor}   {Universidad Politécnica de Victoria}

% NOTA: Dos diagonales juntas (\\) indican un saldo de linea. En este caso particular hay 2 (el titulo se ajusta a tres lineas, porque es muy largo. Hacer las adecuaciones pertinentes
\newcommand{\NombreProyectoheader}     {Notificación de actualizaciones de la información \\en un sistema de información web de pacientes\\ hospitalizados a sus familiares del Sagrado Corazón de Jesus  }
\newcommand{\nasesorempresaria}   {MSI. José Fidencio López Luna}
\newcommand{\fechaPortada}               {Junio de 2021}





\newcommand{\separacionCorta}{0.0cm}
\newcommand{\separacionLarga}{0.5cm}

\usepackage[overload]{textcase}
\newcommand{\iemph}[1]{\MakeTextUppercase{#1}}

\pagestyle{fancy}
\headheight 45pt
\fancyhead{} % Clear all header fields
\fancyhead[L]{\includegraphics[height=1.00cm]{logos/UTYP.png}}%
%\fancyhead[C]{\begin{center}\NombreProyectoheader\end{center}}%
\fancyhead[C]{\begin{center}\scriptsize{\NombreProyectoheader}\end{center}}%
%\fancyhead[R]{\includegraphics[height=1.5cm]{LogoUPV_2019.png}}%
\fancyhead[R]{\includegraphics[height=1.25cm]{logos/LogoUPV_2023.png}}%
\fancyfoot[R]{\thepage} % Clear all footer fields 
\fancyfoot[C]{}
\fancyfoot[L]{}

\DefineBibliographyStrings{english}{%
  references = {Referencias},% replace "references" with "bibliography"  for `book`/`report`
}

\addto\captionsenglish{%
  \renewcommand{\figurename}{Figura}%
  \renewcommand{\tablename}{Tabla}%
} 

\usepackage{wallpaper}
 
 
%\renewcommand{\figurename}{Figura}
%\renewcommand{\tablename}{Tabla}

 
\begin{document}

%-----------------------------------------------------------------------------------------------------------------
% PAGINA 1 - PORTADA
\setcounter{page}{1}
\pagenumbering{roman}
\thispagestyle{empty}

\begin{center}

\begin{tabular}{cp{5cm}c}
\includegraphics[height=2.25cm]{logos/UTYP.png} & 
& \includegraphics[height=2.25cm]{logos/LogoUPV_2023.png}   \\
\end{tabular}

\Large \textbf{UNIVERSIDAD POLITÉCNICA DE VICTORIA}
\vspace{0.5cm}
\hrule
\vspace{0.1cm} 
\hrule
\vspace{0.5cm}


%\HRule \\[\separacionCorta]
\textbf{\iemph{\NombreProyecto}} \\[\separacionLarga]
%\Large \textbf{TESINA}
%\HRule \\[\separacionLarga]
T E S I N A \\
QUE PARA OBTENER EL GRADO DE \\
\textbf{\iemph{\ncarrera}} \\[\separacionLarga]

PRESENTA: \\[\separacionCorta]
%\textbf{\Capitalize{\NombreAlumno}\\[\separacionLarga]
\textbf{\iemph{\NombreAlumno}}\\[\separacionLarga]
%EN CUMPLIMIENTO DE \\[\separacionCorta]
%LA ESTADÍA DE LA CARRERA DE \\[\separacionCorta]


DIRECTOR \\[\separacionCorta]
\textbf{\iemph{\nasesorinstitucional}} \\[\separacionCorta]

CO-DIRECTOR \\[\separacionCorta]
\textbf{\iemph{\nasesorempresaria}} \\[\separacionCorta]

ORGANISMO RECEPTOR \\[\separacionCorta]
\textbf{\iemph{\organismoreceptor}} \\[\separacionLarga]

\end{center}
\begin{flushright}
\iemph{Ciudad Victoria, Tamaulipas, \fechaPortada}
\end{flushright}

\HRule 



% En las siguientes 3 paginas, debe incluir una digitalización de calidad de sus respectivas cartaas, no una fotografia toda cucha tomada con su celular de Coppel
% Pagina 1: Digitalización de la carta de presentación (sustituir por una original en el empastado)
\clearpage
\thispagestyle{fancy}
\pagenumbering{roman}
\setcounter{page}{1}
\includepdf[pages={1}]{cartas/CartaPresentacion.pdf}

% Pagina 2: Digitalización de la carta de aceptación (sustituir por una original en el empastado)
\clearpage
\thispagestyle{fancy}
\includepdf[pages={1}]{cartas/CartaAceptacion.pdf}

% Pagina 3: Digitalización de la carta de liberación (sustituir por una original en el empastado)
\clearpage
\thispagestyle{fancy}
\includepdf[pages={1}]{cartas/CartaLiberacion.pdf}

% Pagina 4: Carta de aceptación del ASESOR INSTITUCIONAL(sustituir por una original en el empastado)
\clearpage
\ULCornerWallPaper{1}{membretes/membrete_UPV_08May25.pdf}
\thispagestyle{empty}

\vspace*{1.5cm}
\large

\begin{center}
\textbf{CARTA DE ACEPTACIÓN DEL DOCUMENTO PARA SU IMPRESIÓN}\\[\separacionLarga]
\end{center}

\begin{flushright}
Cd. Victoria, Tamaulipas a \fechacarta \\[\separacionLarga]
\end{flushright}

\parindent=0mm

\NombreAlumno \\
PRESENTE \\[\separacionCorta]

Le comunico que  el Programa Académico de \ncarrera\ \ le ha otorgado la autorización para la impresión de su Tesina de Estadía Práctica cuyo título es: \\[\separacionCorta]

\begin{center}
\textbf{\NombreProyecto} \\[\separacionLarga]
\end{center}

\begin{center}	
\begin{tabular}{ccc}
\centering
& ATENTAMENTE & \\ 
& & \\
& & \\
& & \\ \hline
& \nasesorinstitucional & \\
& ASESOR INSTITUCIONAL & \\
\end{tabular}
\end{center} 
\vspace{2cm}
c.c.p Director de programa académico


%-----------------------------------------------------------------------------------------------------------------
\clearpage
\ClearWallPaper
\thispagestyle{empty}
\newgeometry{left=1.5cm,top=1.0cm,right=1.5cm,bottom=2.0cm}             
\begin{landscape}

%\afterpage{\restoregeometry}


\begin{tabular}{p{3cm}p{16cm}p{3cm}}
\multirow{4}{*}{\includegraphics[width=3.0cm]{logos/UTYP.png}} &  & \multirow{4}{*}{\includegraphics[width=3.0cm]{logos/LogoUPV_2023.png}} \\
   & \multicolumn{1}{c}{\textbf{EVALUACIÓN DE ESTADÍA}} &  \\ %\cline{2-2}
& \multicolumn{1}{c}{\textbf{Rúbrica para evaluación de la presentación y el reporte de estadía}} & \\ 
& & \\ 
\end{tabular}

\normalsize
\begin{tabular}{p{13cm}R{10cm}}
\multicolumn{1}{l}{Nombre del alumno: \underline{\textbf{\iemph{\NombreAlumno}}}}  & 
Calificación final: \underline{\hspace{3cm}}    \\ %\cline{2-2}
\multicolumn{2}{c}{Periodo: \underline{\textbf{\iemph{\ncuatrimestre}}}}    \\ %\cline{2-2}
& \\
\end{tabular}

\scriptsize 
%\begin{center}
%\resizebox{\linewidth}{!}{%
\begin{tabular}{C{2.0cm}|C{2.0cm}|C{4.5cm}|C{4.5cm}|C{4.5cm}|C{4.5cm}}
\hline
\multirow{2}{*}{Ponderación} & Aspecto a  & Competente & Independiente & Básico Avanzado & No Competente \\
 & Evaluar & 10 & 9 & 8 & 5 \\ 
\hline
40 & Resultados y Actividades & Estrechamente relacionados al perfil de
egreso de su programa académico & Parcialmente relacionados al perfil de
egreso de su programa académico & Escasamente relacionados al perfil de egreso de su programa académico & Escasamente relacionados al perfil de egreso de su programa académico \\  \hline

30 & Exposición de las actividades de la estadía  & Detalladas y sustentadas con respecto a los resultados que se obtuvieron & Detalladas y sustentadas parcialmente con respecto a los resultados que se obtuvieron & Detalladas parcialmente con respecto a los resultados que se obtuvieron & Detalladas escazamente con respecto a los resultados que se obtuvieron \\  \hline 

10 & Material visual Lenguaje
verbal & Uso el lenguaje y la terminología
apropiadas; El material visual está organizado,
adecuado y suficiente
 & 
Uso el lenguaje y la terminología apropiadas
El material visual está parcialmente
organizado y es suficiente
& 
Uso el lenguaje y la terminología son parcialmente apropiadas; El material visual está parcialmente
organizado y es suficiente
 & 
Uso el lenguaje y terminología es inapropiado; El material visual no está organizado y es insuficiente
 \\  \hline
10 & Exposición en Idioma Inglés &

Pronunciation is clear so language is easily understood (2.5) Uses fluent connected speech, occasionally disrupted by search for correct form of expression (2.5) Uses topic related vocabulary without problems (2.5) Responds to questions using varied and descriptive vocabulary and language structures (2.5) & 

Pronunciation is understandable, but there are slight errors (2.25)
Speech is connected but frequently disrupted by search for correct form of
expression (2.25) Uses some topic related vocabulary sufficient to communicate ideas (2.25) Responds to questions using simple but accurate vocabulary and language structures (2.25 & 

Pronunciation is understandable most of the time, marked native accent and many errors
(2) Speaks with simple sentences, sometimes not connected, but is understood (2) Uses basic vocabulary to communicate ideas
(2) Partly responds to simple questions, with limited vocabulary and language structures (2)  &

Pronunciation makes language very difficult to understand (1) Uses one-word/two-word utterances (1) Unable to communicate ideas due to lack of vocabulary (1) Uses isolated words or sentence fragments to respond to questions (1) 
\\  \hline

5 & Respuesta a los cuestionamientos de los evaluadores & 
Clara y satisfactoria & 
Clara y parcialmente satisfactoria & 
Clara e insuficiente & 
Confusa e insuficiente 
\\  \hline
5 & Autorización de tesina en tiempo y forma & 
Presenta en tiempo y forma 
\tikz[overlay, remember picture,anchor=base]  \node (Mark){}; 
 & 
Presenta en tiempo y forma con la mayoría de requerimientos solicitados & 
Presenta en tiempo y con algunas limitantes  de los requerimientos solicitados. &
Presenta fuera de tiempo y con los mínimos requerimientos solicitados.
 \\  % Una linea en Blanco para poner la marca del ASESOR INSTITUCIONAL
&
&
%\tikz[overlay, remember picture,anchor=base] \node (Center){};

&
 \\  \hline

\end{tabular}

%\begin{tikzpicture}[remember picture, overlay, note/.style={rectangle callout, fill=#1}]
%\node [note=red!80, callout absolute pointer={(Mark)}] at (Center) {COMPETENTE!};
%\end{tikzpicture}

%}
%\end{center}

\normalsize


%\begin{center}
%\begin{tabular}{ccc}
%\includegraphics[scale=0.10]{FirmaMANM2.png} & & \\
%\hline 
%\nasesorinstitucional & \nevalador & \\
%ASESOR INSTITUCIONAL & EVALUADOR & EVALUADOR INGLÉS \\
%\end{tabular}
%\end{center}

\begin{center}
\begin{tabular}{cp{1cm}cp{1cm}c}
%\includegraphics[scale=0.10]{ArchivoFirmaASESOR.png} & & & & \\
 & & & & \\
  & & & & \\
\hline 
\nasesorinstitucional & & \nevalador & & \\
ASESOR INSTITUCIONAL & &EVALUADOR & &EVALUADOR DE INGLÉS \\
\end{tabular}
\end{center}


\end{landscape} 
\restoregeometry

%-----------------------------------------------------------------------------------------------------------------
% Pagina 5: Registro de Evaluación de Estadía
\clearpage
\pagestyle{empty}
\ULCornerWallPaper{1}{membretes/membrete_UPV_08May25.pdf}

\vspace*{0.5cm}



\large
\begin{center}
\textbf{REGISTRO DE EVALUACIÓN DE EXPOSICIÓN DE ESTADÍA}
\\[\separacionLarga]
\end{center}


Siendo las \HoraExposionFormatoVenticuatroHoras \ \ horas del día \FechaExposicion, \elolaNombreAlumno\ \ alumn\OA\ \ \textbf{\NombreAlumno}, del programa académico \textbf{\ncarrera}, con matricula \textbf{\Matricula}, presentó la exposición de la estadía realizada durante el cuatrimestre \textbf{\ncuatrimestre}, en \elolaNombreEmpresa\ \ \textbf{\organismoreceptor}, con el proyecto titulado \textbf{\NombreProyecto}.\\

Una vez concluido el proceso de evaluación, y con base a la rúbrica establecida para éste propósito, se determina que la calificación de la estadía es \underline{\hspace{2cm}}. %\hrulefill.

\begin{center}
\begin{tabular}{ccc}
& & \\
& & \\
& & \\
%& \includegraphics[scale=0.10]{FirmaAsesorPro.png} & \\
\hline 
%& \underline{\hspace{8cm}}& \\
& \nasesorinstitucional & \\
& ASESOR INSTITUCIONAL & \\
& & \\
& & \\
& & \\
%& \underline{\hspace{8cm}}& \\
\hline 
& \nevalador & \\
& EVALUADOR & \\
& & \\
& & \\
%& & \\
\hline 
& & \\
& EVALUADOR DE INGLÉS & \\
\end{tabular}
\end{center}

\normalsize

%-----------------------------------------------------------------------------------------------------------------
% PAGINA 4 - AGRADECIMIENTOS
\clearpage
\ClearWallPaper
\pagestyle{fancy}
\section*{\centering Agradecimientos}
\addcontentsline{toc}{section}{Agradecimientos}
%\input{Agradecimientos.tex}

%-----------------------------------------------------------------------------------------------------------------
% PAGINA 5 - RESUMEN EN ESPAÑOL

\clearpage
\section*{\centering Resumen}
\addcontentsline{toc}{section}{Resumen}
%\input{Resumen.tex}
\textbf{Palabras clave:} Tecnología, Sistema web, Desarrollar, Implementar, Empresas, Modulo.

%-----------------------------------------------------------------------------------------------------------------
% PAGINA 6 - RESUMEN EN INGLES

\clearpage
\section*{\centering Summary}
\addcontentsline{toc}{section}{Summary}
%\input{Summary.tex}
\textbf{Keywords}: Technology, Web system, Develop, Implement, Companies, Module.

%-----------------------------------------------------------------------------------------------------------------
% PAGINA 7 - INDICE

\clearpage
\addcontentsline{toc}{section}{Índice}
\renewcommand\contentsname{Índice}
\tableofcontents

%-----------------------------------------------------------------------------------------------------------------
% CAPITULOS


\clearpage
\pagenumbering{arabic}
\setcounter{page}{1}
%\input{Capitulo1.tex}

\clearpage
%\input{Capitulo2.tex}

\clearpage
\section{Introducción}
\label{ch:introduccion}

La introducción es la primera sección sustantiva de la tesis y tiene como propósito principal guiar al lector a través del contexto y la relevancia del estudio. En ella, se presenta el problema de investigación, se establecen los objetivos, se justifica la realización del trabajo y se delinean los alcances y limitaciones. Debe ser escrita de forma clara y concisa, capturando el interés del lector y proporcionando una visión general de lo que se abordará en los capítulos subsiguientes.

\subsection{Antecedentes}
\label{sec:antecedentes}

En esta sección, se presenta una revisión concisa y pertinente de investigaciones previas, literatura existente y el estado del arte relacionado con el tema de la tesis. El objetivo es situar el trabajo actual dentro del conocimiento acumulado, identificar vacíos en la investigación y demostrar la originalidad y la necesidad del estudio propuesto. Se deben citar adecuadamente las fuentes, mostrando cómo el trabajo se construye sobre cimientos previos o cómo aborda aspectos no resueltos.

\subsection{Definición del Problema}
\label{sec:problema}

Aquí se describe de manera clara y precisa el problema que la tesis busca resolver. Es fundamental articular el problema de forma que sea comprensible, relevante y factible de investigar. Esta sección debe responder a preguntas como: ¿Qué situación actual es insatisfactoria o genera una necesidad? ¿Qué aspectos de este problema no han sido abordados adecuadamente por investigaciones anteriores? La definición del problema puede ser formulada como una pregunta de investigación o una declaración clara de la brecha de conocimiento.

\subsection{Objetivos}
\label{sec:objetivos}

Los objetivos son las metas que se esperan alcanzar con la realización de la investigación. Deben ser claros, medibles, alcanzables, relevantes y con un tiempo definido (SMART, por sus siglas en inglés). Se dividen en un objetivo general y varios objetivos específicos.

\subsubsection{Objetivo General}
\label{subsec:objetivo_general}

El objetivo general expresa el propósito fundamental del estudio en términos amplios. Es la meta principal que se busca lograr y generalmente abarca la totalidad de la investigación. Debe ser coherente con la definición del problema.
\begin{itemize}
    \item \textbf{Ejemplo:} Desarrollar un sistema automatizado para la detección temprana de fallas en turbinas eólicas utilizando técnicas de aprendizaje automático.
\end{itemize}

\subsubsection{Objetivos Específicos}
\label{subsec:objetivos_especificos}

Los objetivos específicos detallan los pasos o las acciones concretas que se llevarán a cabo para alcanzar el objetivo general. Son tareas más pequeñas y desglosadas que, al ser cumplidas, permiten la consecución del objetivo principal.
\begin{itemize}
    \item \textbf{Ejemplo 1:} Recopilar y preprocesar un conjunto de datos de vibraciones y parámetros operativos de turbinas eólicas.
    \item \textbf{Ejemplo 2:} Implementar y evaluar diferentes algoritmos de aprendizaje automático para la clasificación de estados operativos (normal, pre-falla, falla).
    \item \textbf{Ejemplo 3:} Diseñar una interfaz de usuario para la visualización de alertas y el monitoreo en tiempo real del estado de las turbinas.
\end{itemize}

\subsection{Justificación}
\label{sec:justificacion}

La justificación explica por qué es importante y relevante llevar a cabo la investigación. Se deben argumentar los beneficios y la utilidad del estudio desde diferentes perspectivas: teórica, práctica, social, económica, tecnológica, etc. Responde a la pregunta: ¿Por qué vale la pena invertir tiempo y recursos en esta investigación? Aquí se resaltan los aportes que la tesis generará al campo del conocimiento o a la solución de problemas específicos.

\subsection{Alcances y Limitaciones}
\label{sec:alcances_limitaciones}

\subsubsection{Alcances}
\label{subsec:alcances}

Los alcances definen claramente hasta dónde llegará el estudio, qué aspectos cubrirá y cuáles serán sus fronteras. Establecen qué elementos, variables, poblaciones o contextos serán incluidos en la investigación. Es esencial para evitar expectativas irrealistas y para mantener el enfoque del trabajo.

\subsubsection{Limitaciones}
\label{subsec:limitaciones}

Las limitaciones son los factores externos o internos que podrían restringir la validez, generalización o profundidad de los resultados de la investigación. Es importante identificarlas honestamente, ya que demuestran un pensamiento crítico por parte del investigador. Pueden incluir restricciones de tiempo, recursos, acceso a datos, metodológicas, etc.

\subsection{Ejemplos de Tablas e Inclusión de Imágenes}
\label{sec:ejemplos_elementos}

Para ilustrar el uso de elementos visuales en su tesis, a continuación se presentan ejemplos básicos de cómo incluir tablas e imágenes en LaTeX.

\subsubsection{Inclusión de Tablas}

Las tablas son herramientas fundamentales para presentar datos de manera organizada y comprensible.
\begin{table}[H]
    \centering
    \caption{Clasificación de Sensores Comunes en Ingeniería.}
    \label{tab:sensores_comunes}
    \begin{tabular}{|l|c|c|}
        \hline
        \textbf{Tipo de Sensor} & \textbf{Magnitud Medida} & \textbf{Principio de Funcionamiento} \\
        \hline
        Termopar & Temperatura & Efecto Seebeck \\
        Extensómetro & Deformación & Resistencia eléctrica \\
        Acelerómetro & Aceleración & Efecto piezoeléctrico \\
        Encoder rotatorio & Posición angular & Óptico/Magnético \\
        \hline
    \end{tabular}
    \caption*{Nota: Esta tabla es un ejemplo para demostrar la estructura básica de una tabla en LaTeX.}
\end{table}
Como se puede observar en la Tabla \ref{tab:sensores_comunes}, la información se presenta de forma clara y estructurada. Es importante añadir un \verb|`\caption`|  para describir la tabla y un  \verb|`\label`|  para referenciarla en el texto.

\subsubsection{Inclusión de Imágenes}

Las imágenes, gráficos o diagramas son cruciales para complementar la explicación textual y facilitar la comprensión de conceptos complejos o resultados visuales.
\begin{figure}[H]
    \centering
    \includegraphics[width=0.7\textwidth]{img/diagrama_proceso}
    \caption{Diagrama de un proceso de control automático.}
    \label{fig:diagrama_control}
\end{figure}
% La Figura \ref{fig:diagrama_control} ilustra un ejemplo de un diagrama de un proceso de control. Para incluir una imagen, se utiliza el entorno `figure` y el comando `\includegraphics`. Es fundamental especificar la ruta de la imagen (en este caso, `images/diagrama_proceso.png`) y un `width` para controlar su tamaño. Al igual que con las tablas, un `\caption` y un `\label` son indispensables. Es crucial que cada imagen tenga una calidad adecuada y que sea relevante para el texto.

\begin{figure}[H]
    \centering
    \begin{subfigure}[b]{0.45\textwidth}
        \includegraphics[width=\textwidth]{img/grafico_datos1}
        \caption{Datos de Temperatura.}
        \label{fig:subfig_temp}
    \end{subfigure}
    \hfill
    \begin{subfigure}[b]{0.45\textwidth}
        \includegraphics[width=\textwidth]{img/grafico_datos2}
        \caption{Datos de Presión.}
        \label{fig:subfig_pres}
    \end{subfigure}
    \caption{Comparación de datos operativos de un sistema (Temperatura vs. Presión).}
    \label{fig:comparacion_datos}
\end{figure}
Incluso es posible combinar varias imágenes en una sola figura, como se muestra en la Figura \ref{fig:comparacion_datos}, utilizando el paquete `subcaption`.

Asegúrense de que todas las figuras y tablas estén debidamente referenciadas en el texto y que su contenido sea autoexplicativo, utilizando los pies de figura y títulos de tabla de manera efectiva.

\clearpage
% ======================================================================
% Archivo: capitulo2.tex
% Este archivo contiene el contenido del Capítulo 2: Estado del Arte
% o Trabajos Relacionados.
% ======================================================================
\section{Estado del Arte y Trabajos Relacionados}
\label{ch:estado-del-arte}

Este capítulo tiene como objetivo establecer el contexto de la investigación mediante la revisión de la literatura científica y técnica existente. Se exponen los trabajos previos más relevantes, identificando sus aportaciones, metodologías y resultados, lo que permite justificar la originalidad y pertinencia de la presente tesis o tesina.

% ----------------------------------------------------------------------
\subsection{El Estado del Arte}
\label{sec:estado-del-arte}

El **Estado del Arte** es una revisión exhaustiva y crítica de la literatura académica y técnica disponible sobre un tema de investigación. Su propósito no es solo resumir lo que ya se ha hecho, sino también identificar:
\begin{itemize}
    \item La **evolución histórica** del problema.
    \item Los **conceptos y teorías clave** que sustentan el campo.
    \item Las **tendencias actuales** y las líneas de investigación más activas.
    \item Las **brechas de conocimiento** o preguntas sin resolver, lo que justifica la necesidad de tu propia investigación.
\end{itemize}
En esta sección, se busca proporcionar un panorama general del campo de estudio, demostrando que comprendes el marco teórico y contextual de tu proyecto.

% ----------------------------------------------------------------------
\subsection{Trabajos Relacionados}
\label{sec:trabajos-relacionados}

Los **Trabajos Relacionados** se centran en un conjunto más selecto de investigaciones que están directamente vinculadas con el problema específico que abordas en tu tesis. A diferencia del Estado del Arte, que es más amplio, los Trabajos Relacionados buscan:
\begin{itemize}
    \item Describir los **métodos y técnicas** específicas utilizadas por otros autores.
    \item Analizar las **ventajas y limitaciones** de sus soluciones.
    \item Identificar en qué medida sus trabajos se solapan o difieren de tu propuesta.
    \item Sentar las bases para la **justificación de tu enfoque**. Por ejemplo, si un trabajo previo tiene una limitación clara (baja precisión, alto costo computacional, etc.), tu tesis puede proponer una solución para superarla.
\end{itemize}
Se recomienda organizar esta sección de manera temática o cronológica para facilitar la comprensión.

% ----------------------------------------------------------------------
\subsection{Ejemplos de Tablas}
\label{sec:ejemplos-tablas}

Las tablas son herramientas visuales poderosas para resumir información. Dependiendo de tu objetivo, puedes utilizar una tabla informativa o una tabla comparativa.

% --- Tabla Informativa ---
\subsubsection{Tabla Informativa}
\label{subsec:tabla-informativa}
Una **tabla informativa** se utiliza para resumir y presentar los detalles clave de varios trabajos de manera concisa. Es ideal para el inicio del capítulo o para el "Estado del Arte", cuando necesitas dar un panorama general de los trabajos consultados. El enfoque es describir "qué es" cada trabajo.

\begin{table}[H]
    \centering
    \caption{Ejemplo de Tabla Informativa de Trabajos Previos}
    \label{tab:informativa}
    \begin{tabular}{|p{2cm}|p{2cm}|p{3cm}|p{5cm}|}
        \hline
        \textbf{Autor (Año)} & \textbf{Tema} & \textbf{Metodología} & \textbf{Contribución Clave} \\
        \hline
        Pérez (2018) & Detección de grietas en concreto. & Visión artificial y procesamiento de imágenes. & Desarrolló un algoritmo para identificar grietas con 90\% de precisión. \\
        \hline
        Gómez y López (2019) & Monitoreo de puentes. & Sensores de vibración y análisis de datos. & Propusieron un sistema de alerta temprana para fallas estructurales. \\
        \hline
        Martínez (2020) & Optimización de rutas logísticas. & Algoritmos genéticos y modelado de red. & Redujo los tiempos de entrega en 15\% para un caso de estudio. \\
        \hline
    \end{tabular}
\end{table}

% --- Tabla Comparativa ---
\subsubsection{Tabla Comparativa}
\label{subsec:tabla-comparativa}
Una **tabla comparativa** se utiliza para el análisis crítico. Su propósito es comparar directamente los trabajos en función de criterios específicos que son relevantes para tu investigación. Es más adecuada para la sección de "Trabajos Relacionados" y te ayuda a justificar por qué tu enfoque es mejor o diferente. El enfoque es "cómo se compara" cada trabajo.

\begin{table}[H]
    \centering
    \caption{Ejemplo de Tabla Comparativa de Algoritmos}
    \label{tab:comparativa}
    \begin{tabular}{|p{2.5cm}|c|c|c|c|}
        \hline
        \textbf{Algoritmo/Autor} & \textbf{Precisión (\%)} & \textbf{Velocidad (ms)} & \textbf{Escalabilidad} & \textbf{Recursos de HW} \\
        \hline
        Pérez (2018) & 90\% & 50 & Baja & PC estándar \\
        \hline
        Gómez y López (2019) & 95\% & 200 & Media & Servidor de alto rendimiento \\
        \hline
        **Propuesta (Tesis)** & **98\%** & **35** & **Alta** & **PC estándar** \\
        \hline
    \end{tabular}
\end{table}

% ----------------------------------------------------------------------
\subsection{Diferencia entre Tablas y Cuándo Utilizarlas}

---

La principal diferencia radica en su **propósito y enfoque**:

* **Tabla Informativa:** Es **descriptiva**. Su objetivo es organizar la información básica de múltiples fuentes para que el lector pueda obtener una visión general rápida. Úsala cuando quieras listar los principales trabajos en el campo de manera ordenada.
* **Tabla Comparativa:** Es **analítica y evaluativa**. Su objetivo es contrastar características, métricas de rendimiento o metodologías específicas para destacar las ventajas y desventajas de cada trabajo. Úsala para justificar por qué tu propuesta de tesis es necesaria y cómo mejora o difiere de lo que ya existe. Es crucial para demostrar la originalidad de tu trabajo.

**Para tus alumnos:** Aconseja que utilicen una **tabla comparativa** si el tema de su tesis se centra en la mejora de un sistema existente, un algoritmo, o una metodología, ya que les permitirá demostrar con datos por qué su solución es superior. Si la tesis es más teórica o conceptual, una **tabla informativa** será suficiente para contextualizar el trabajo.

Recuerda que estas tablas no sustituyen la descripción en texto. La explicación narrativa de cada trabajo es fundamental para un análisis completo y riguroso.

% Fin del capítulo 2
% ======================================================================

\clearpage
\addcontentsline{toc}{section}{Índice de figuras}
\renewcommand\listfigurename{Índice de figuras}

\listoffigures

\clearpage
\addcontentsline{toc}{section}{Índice de cuadros}
\renewcommand\listtablename{Índice de cuadros}
\listoftables

\clearpage
\addcontentsline{toc}{section}{Índice de algoritmos}
\renewcommand\listalgorithmname{Índice de algoritmos}
\listofalgorithms

%-----------------------------------------------------------------------------------------------------------------
% REFERENCIAS


\clearpage
%Let's cite! The Einstein's journal paper \cite{dirac} and the Dirac's 
%book \cite{einstein} are physics related items. 

%\Urlmuskip=0mu plus 1mu\relax
\addcontentsline{toc}{section}{Referencias} 
\printbibliography
 
\end{document}
