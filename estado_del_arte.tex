% ======================================================================
% Archivo: capitulo2.tex
% Este archivo contiene el contenido del Capítulo 2: Estado del Arte
% o Trabajos Relacionados.
% ======================================================================
\section{Estado del Arte y Trabajos Relacionados}
\label{ch:estado-del-arte}

Este capítulo tiene como objetivo establecer el contexto de la investigación mediante la revisión de la literatura científica y técnica existente. Se exponen los trabajos previos más relevantes, identificando sus aportaciones, metodologías y resultados, lo que permite justificar la originalidad y pertinencia de la presente tesis o tesina.

% ----------------------------------------------------------------------
\subsection{El Estado del Arte}
\label{sec:estado-del-arte}

El **Estado del Arte** es una revisión exhaustiva y crítica de la literatura académica y técnica disponible sobre un tema de investigación. Su propósito no es solo resumir lo que ya se ha hecho, sino también identificar:
\begin{itemize}
    \item La **evolución histórica** del problema.
    \item Los **conceptos y teorías clave** que sustentan el campo.
    \item Las **tendencias actuales** y las líneas de investigación más activas.
    \item Las **brechas de conocimiento** o preguntas sin resolver, lo que justifica la necesidad de tu propia investigación.
\end{itemize}
En esta sección, se busca proporcionar un panorama general del campo de estudio, demostrando que comprendes el marco teórico y contextual de tu proyecto.

% ----------------------------------------------------------------------
\subsection{Trabajos Relacionados}
\label{sec:trabajos-relacionados}

Los **Trabajos Relacionados** se centran en un conjunto más selecto de investigaciones que están directamente vinculadas con el problema específico que abordas en tu tesis. A diferencia del Estado del Arte, que es más amplio, los Trabajos Relacionados buscan:
\begin{itemize}
    \item Describir los **métodos y técnicas** específicas utilizadas por otros autores.
    \item Analizar las **ventajas y limitaciones** de sus soluciones.
    \item Identificar en qué medida sus trabajos se solapan o difieren de tu propuesta.
    \item Sentar las bases para la **justificación de tu enfoque**. Por ejemplo, si un trabajo previo tiene una limitación clara (baja precisión, alto costo computacional, etc.), tu tesis puede proponer una solución para superarla.
\end{itemize}
Se recomienda organizar esta sección de manera temática o cronológica para facilitar la comprensión.

% ----------------------------------------------------------------------
\subsection{Ejemplos de Tablas}
\label{sec:ejemplos-tablas}

Las tablas son herramientas visuales poderosas para resumir información. Dependiendo de tu objetivo, puedes utilizar una tabla informativa o una tabla comparativa.

% --- Tabla Informativa ---
\subsubsection{Tabla Informativa}
\label{subsec:tabla-informativa}
Una **tabla informativa** se utiliza para resumir y presentar los detalles clave de varios trabajos de manera concisa. Es ideal para el inicio del capítulo o para el "Estado del Arte", cuando necesitas dar un panorama general de los trabajos consultados. El enfoque es describir "qué es" cada trabajo.

\begin{table}[H]
    \centering
    \caption{Ejemplo de Tabla Informativa de Trabajos Previos}
    \label{tab:informativa}
    \begin{tabular}{|p{2cm}|p{2cm}|p{3cm}|p{5cm}|}
        \hline
        \textbf{Autor (Año)} & \textbf{Tema} & \textbf{Metodología} & \textbf{Contribución Clave} \\
        \hline
        Pérez (2018) & Detección de grietas en concreto. & Visión artificial y procesamiento de imágenes. & Desarrolló un algoritmo para identificar grietas con 90\% de precisión. \\
        \hline
        Gómez y López (2019) & Monitoreo de puentes. & Sensores de vibración y análisis de datos. & Propusieron un sistema de alerta temprana para fallas estructurales. \\
        \hline
        Martínez (2020) & Optimización de rutas logísticas. & Algoritmos genéticos y modelado de red. & Redujo los tiempos de entrega en 15\% para un caso de estudio. \\
        \hline
    \end{tabular}
\end{table}

% --- Tabla Comparativa ---
\subsubsection{Tabla Comparativa}
\label{subsec:tabla-comparativa}
Una **tabla comparativa** se utiliza para el análisis crítico. Su propósito es comparar directamente los trabajos en función de criterios específicos que son relevantes para tu investigación. Es más adecuada para la sección de "Trabajos Relacionados" y te ayuda a justificar por qué tu enfoque es mejor o diferente. El enfoque es "cómo se compara" cada trabajo.

\begin{table}[H]
    \centering
    \caption{Ejemplo de Tabla Comparativa de Algoritmos}
    \label{tab:comparativa}
    \begin{tabular}{|p{2.5cm}|c|c|c|c|}
        \hline
        \textbf{Algoritmo/Autor} & \textbf{Precisión (\%)} & \textbf{Velocidad (ms)} & \textbf{Escalabilidad} & \textbf{Recursos de HW} \\
        \hline
        Pérez (2018) & 90\% & 50 & Baja & PC estándar \\
        \hline
        Gómez y López (2019) & 95\% & 200 & Media & Servidor de alto rendimiento \\
        \hline
        **Propuesta (Tesis)** & **98\%** & **35** & **Alta** & **PC estándar** \\
        \hline
    \end{tabular}
\end{table}

% ----------------------------------------------------------------------
\subsection{Diferencia entre Tablas y Cuándo Utilizarlas}

---

La principal diferencia radica en su **propósito y enfoque**:

* **Tabla Informativa:** Es **descriptiva**. Su objetivo es organizar la información básica de múltiples fuentes para que el lector pueda obtener una visión general rápida. Úsala cuando quieras listar los principales trabajos en el campo de manera ordenada.
* **Tabla Comparativa:** Es **analítica y evaluativa**. Su objetivo es contrastar características, métricas de rendimiento o metodologías específicas para destacar las ventajas y desventajas de cada trabajo. Úsala para justificar por qué tu propuesta de tesis es necesaria y cómo mejora o difiere de lo que ya existe. Es crucial para demostrar la originalidad de tu trabajo.

**Para tus alumnos:** Aconseja que utilicen una **tabla comparativa** si el tema de su tesis se centra en la mejora de un sistema existente, un algoritmo, o una metodología, ya que les permitirá demostrar con datos por qué su solución es superior. Si la tesis es más teórica o conceptual, una **tabla informativa** será suficiente para contextualizar el trabajo.

Recuerda que estas tablas no sustituyen la descripción en texto. La explicación narrativa de cada trabajo es fundamental para un análisis completo y riguroso.

% Fin del capítulo 2
% ======================================================================